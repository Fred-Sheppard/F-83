%! Author = frede
%! Date = 19/11/2022

% Preamble
\documentclass[11pt]{article}
\newcommand{\Rnd}{\text{Rnd}}
\newcommand{\Pol}{\text{Pol}}
\newcommand{\Rec}{\text{Rec}}
\newcommand{\Perm}[2] { {}^{#1} \text{P}_{#2} }

% Packages
\usepackage{amsmath}



% Document
\begin{document}
    \title{F-83 Scripting Manual}
    \author{Fred Sheppard}
    \maketitle{}
    \pagebreak


    \section{Decimal to Binary}\label{sec:decimal-to-binary}

    \begin{align}
        &\frac{0}{y} + 10^x \left(y-2\Rnd\left(\frac{y}{2}-0.5\right)\right) + \\
        &0\Pol\left[\left(x+1\right) * \cos \left(\Rnd\left(\frac{y}{2}-0.5\right)\right),
        \left(x+1\right) * \sin \left(\Rnd\left(\frac{y}{2}-0.5\right)\right)\right] M+
    \end{align}

    \begin{center}
        \begin{tabular}{|l|l|l|}
            \hline
            Variable & Start & End  \\
            \hline
            $x$      & 0     & ?    \\
            \hline
            $y$      & 10    & 0    \\
            \hline
            $M$      & 0     & 1010 \\
            \hline
        \end{tabular}
    \end{center}


    \section{Binary to Decimal}\label{sec:binary-to-decimal}
    \begin{align}
        &\frac {0} { x - \Rnd\left( \log \left( A \right) + 0.5 \right) } +
        2^x \left[ \Rnd \left( \frac{A} {10^x} - 0.5 \right) -10\Rnd \left( \frac{A}{10^{x+1}}-0.5 \right)\right] \\
        &+ 0\Rec \left( x+1, 0 \right) M+
    \end{align}

    \begin{center}
        \begin{tabular}{|l|l|l|}
            \hline
            Variable & Start & End \\
            \hline
            $x$      & 0     & ?   \\
            \hline
            $A$      & 1011  & 0   \\
            \hline
            $M$      & 0     & 11  \\
            \hline
        \end{tabular}
    \end{center}


    \section{Sum}\label{sec:sum}
    \begin{align}
        \sum_{x=a}^{n} f\left(x\right) =
        f(x) + \frac{0\Rec\left( x+1,~0\right)} {x -n -2} M+
    \end{align}

    \begin{center}
        \begin{tabular}{|l|l|l|}
            \hline
            Variable & Start & End      \\
            \hline
            $x$      & $a$   & n        \\
            \hline
            $M$      & 0     & $\Sigma$ \\
            \hline
        \end{tabular}
    \end{center}


    \section{Sequences}\label{sec:sequences}
    \begin{align}
        Tn &=  A \cdot r^n \\
        r &=  {\frac {T_y} {T_x} } ^ { \left( y-x \right) ^ {-1} }  \\
        A &= \frac {T_x} {r^x}  \\
        Sn &= \frac{ Ar \left( 1 - r^n \right) } { 1 - r }
    \end{align}

    Where A is the starting value i.e. $T_0$ and r is the common ratio.

    In 2009, the population was 2,000.
    By 2013, the population was 32,000.
    Find the general formula.


    \section{Probability}\label{sec:probability}

    \paragraph{Venn Diagram}
    \begin{align}
        P\left(E \cup F\right) = P\left(E\right) + P\left(F\right) - P\left(E \cap F\right)
    \end{align}

    \paragraph{Conditional Events}
    \begin{align}
        P\left(F|G\right) = \frac{P\left(F \cap G\right) } { P\left(G\right) }
    \end{align}

    \paragraph{Mutually Exclusive}
    Both events cannot occur simultaneously - Their intersection is an empty set. \\
    Draw a single card that: Is black \& is a Jack of Diamonds
    \begin{align}
        P\left(E \cup F\right) = P\left(E\right) + P\left(F\right)
    \end{align}

    \paragraph{Independent Events}
    The outcome of one event has no bearing on the other.
    Roll a die and flip a coin.
    \begin{align}
        P\left(E \cap F\right) = P\left(E\right) \cdot P\left(F\right)
    \end{align}

    \paragraph{Selection: Order doesn't matter}
    There are 23 balls in a box: 12 red, 6 blue and 5 green.
    3 balls are chosen at random from the box.
    What is the probability they are all different colours?
    \begin{align}
        P = \frac{ \binom{12}{1} \binom{6}{1} \binom{5}{1} } {\binom{23}{3}}
    \end{align}

    What is the probability exactly two balls are red?
    \begin{align}
        P = \frac{ \binom{12}{2} \binom{11}{1} } {\binom{23}{3}}
    \end{align}

    The tops and bottoms of the numerator must sum to the tops and bottoms
    of the denominator.

    \paragraph{Selection: Order Matters}
    Four balls are selected from the box above.
    What is the probability that the first 3 will be red
    and the 4th will be any other colour?
    \begin{align}
        P = \frac { \Perm{12}{3} \cdot \Perm{11}{1} } {\Perm{23}{4}}
    \end{align}

\end{document}
