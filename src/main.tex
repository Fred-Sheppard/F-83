%! Author = frede
%! Date = 19/11/2022

% Preamble
\documentclass[11pt]{article}
\newcommand{\Rnd}{\text{Rnd}}
\newcommand{\Pol}{\text{Pol}}
\newcommand{\Rec}{\text{Rec}}

% Packages
\usepackage{amsmath}

% Document
\begin{document}
    \title{F-83 Scripting Manual}
    \author{Fred Sheppard}
    \maketitle{}
    \pagebreak


    \section{Decimal to Binary}\label{sec:decimal-to-binary}

    \begin{align}
        &\frac{0}{y} + 10^x \left(y-2\Rnd\left(\frac{y}{2}-0.5\right)\right) + \\
        &0\Pol\left[\left(x+1\right) * \cos \left(\Rnd\left(\frac{y}{2}-0.5\right)\right),
        \left(x+1\right) * \sin \left(\Rnd\left(\frac{y}{2}-0.5\right)\right)\right] M+
    \end{align}

    \begin{center}
        \begin{tabular}{|l|l|l|}
            \hline
            Variable & Start & End  \\
            \hline
            $x$      & 0     & ?    \\
            \hline
            $y$      & 10    & 0    \\
            \hline
            $M$      & 0     & 1010 \\
            \hline
        \end{tabular}
    \end{center}


    \section{Binary to Decimal}\label{sec:binary-to-decimal}
    \[
        \frac {0} { x - \Rnd\left( \log \left( A \right) + 0.5 \right) } +
        2^x \left[ \Rnd \left( \frac{A} {10^x} - 0.5 \right) -10\Rnd \left( \frac{A}{10^{x+1}}-0.5 \right)\right]
        + 0\Rec \left( x+1, 0 \right) M+
    \]


    \begin{center}
        \begin{tabular}{|l|l|l|}
            \hline
            Variable & Start & End \\
            \hline
            $x$      & 0     & ?   \\
            \hline
            $A$      & 1011  & 0   \\
            \hline
            $M$      & 0     & 11  \\
            \hline
        \end{tabular}
    \end{center}


    \section{Sum}\label{sec:sum}
    \[
        \sum_{x=a}^{n} f\left(x\right) =
        f(x) + \frac{0\Rec\left( x+1,~0\right)} {x -n -2} M+
    \]


    \begin{center}
        \begin{tabular}{|l|l|l|}
            \hline
            Variable & Start & End      \\
            \hline
            $x$      & $a$   & n        \\
            \hline
            $M$      & 0     & $\Sigma$ \\
            \hline
        \end{tabular}
    \end{center}


    \section{Sequences}\label{sec:sequences}
    \begin{align}
        Tn &=  A \cdot r^n \\
        r &=  {\frac {T_y} {T_x} } ^ { \left( y-x \right) ^ {-1} }  \\
        A &= \frac {T_x} {r^x}  \\
        Sn &= \frac{ Ar \left( 1 - r^n \right) } { 1 - r }
    \end{align}

    Where A is the starting value i.e. $T_0$ and r is the common ratio.


    In 2009, the population was 2,000.
    By 2013, the population was 32,000.
    Find the general formula.


    \section{Financial Maths}\label{sec:financial-maths}
    \paragraph{Mortgage}
    \begin{align}
        A &= P \cdot \frac {i} { 1- \left(1+i\right) ^{-t}}
    \end{align}

    \paragraph{Investment}
    \begin{align}
        A &= -P \cdot \frac {i} { 1- \left(1-i\right) ^{-t}}
    \end{align}

    Where: \\
    $A$ is the monthly payment \\
    $P$ is the Principle \\
    $i$ is the monthly interest rate \\
    $t$ is the time period in months \\

    Note the two swapped signs in the Investment formula. \\
    To get Principle from monthly payment, simply swap A \& P, and invert the fraction.

\end{document}
